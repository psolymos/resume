%\section{Courses I would like to teach} % ------------------------

\subsection{Management and analysis of ecological data} % ------------------

Ecology is becoming a data rich discipline, thus data exploration and visualization is more and more important. This often requires advance data manipulation. The course would introduce basic database concepts, best practices for data manipulation, exploratory and explanatory analysis:
\vspace{0.3pc}

\begin{itemize}%
\item[]% Empty item
\begin{itemize}%
  \item Common types of data (compositional, continuous measurements, counts, proportions, discrete and ordinal data, etc.); 
  \item Database representations (relational, non-relational); 
  \item Keys, schemas, relational operations, queries, pivot tables; 
  \item Data exploration/visualization (data summaries, plots, histograms); 
  \item Finding groups in the data (clustering); 
  \item Finding gradients in the data (ordination); 
  \item Basic concepts in statistical inference and prediction; 
  \item Overview of parametric (GLM), semi-parametric (GAM), non-parametric (regression trees) methods;
\end{itemize}%
\end{itemize}%
\vspace{0.5pc}

I have teaching (undergraduate and graduate) and communication experience with diverse audiences (government, industry, consulting, academia). I have expertise in manipulating and analyzing large scale data sets and have developed R packages directly relevant to this course: \textbf{mefa} and \textbf{mefa4} for data management, \textbf{vegan} for data visualization and multivariate methods. Being a software developer also helps me in teaching others about statistical techniques. 
\vspace{0.5pc}

\subsection{Ecoinformatics}
I am an expert in management and analysis of large data sets and in developing methods for complex hierarchical models. Being an ecologist with 10+ years of experience of collaborative research involving complex data and analysis problems with on-the-ground conservation and management relevance, I am well positioned to teach a graduate level course in ecoinformatics.
Textbook examples are often focused on small data sets for didactical reasons. The key in this course would be using real-world ecological data sets for real world problems. The focus would be on helping students familiarizing themselves with concepts and (free and open source) software environments necessary for effectively dealing with large data sets and complex problems. I can see two complementary modules that would address different aspects of the challenges:
\vspace{0.3pc}

\begin{itemize}%
\item High performance computing (HPC):
\begin{itemize}%
  \item State of the art in HPC: sparse matrices, parallelism on multicore machines (``socket'' clusters, forking), grid computing, shared memory applications;
  \item Distributed data processing (map-reduce);
  \item File transfer protocols for distributed computing;
  \item Overview of ``embarrassingly parallel problems'' in ecology;
  \item Implementing HPC procedures on computing grids;
  \item Best coding practices: vectorization, memory pre-allocation, etc.
\end{itemize}%
\item Markov-chain Monte Carlo (MCMC) methods:
\begin{itemize}%
  \item Algorithms for sampling unknown distributions; 
  \item Bayesian methodology, the BUGS language;
  \item Maximum likelihood and Bayesian inference: a comparison;
  \item The data cloning algorithm;
  \item Overview of hierarchical models in ecology;
  \item Implementing Bayesian models;
  \item Best coding practices in JAGS.
\end{itemize}%
\end{itemize}%
\vspace{0.5pc}

\subsection{Quantitative methods in ecology}
Statistical literacy is a key for success in ecology. My research and software developer experience positions me well in teaching quantitative methods. Methods that I would feel comfortable teaching include:
\vspace{0.3pc}

\begin{itemize}%
\item[]% Empty item
\begin{itemize}%
  \item Presence-only data (animal movement, resource selection);
  \item Zero inflated count data;
  \item Correcting for detection error (removal, double observer, distance sampling, etc.);
  \item Trend analysis, population viability analysis;
  \item Meta-analysis;
  \item Variable selection with regularization (LASSO, ridge regression, elastic net penalty);
  \item Machine learning techniques (regression trees, boosting, bagging etc.).
\end{itemize}%
\end{itemize}%
\vspace{0.5pc}

\subsection{Biogeography/Macroecology}
Classical biogeography and paleoecology, the spatial and temporal aspects of biodiversity, have always fascinated me since I was a student. These disciplines provide the ``big picture'' for conservation actions made at local scales. Recent advances in the field of macroecology include phylogeography, species distribution modeling besides the classical themes in island biogeography and hotspot analysis. I am actively involved in macroecological analysis, and given my previous teaching experience in ecology, conservation biology, and biogeography, I would really enjoy developing a course in biogeography and macroecology.
\vspace{0.5pc}

\subsection{Science communication: writing and presenting research effectively}
Proposals, papers and presentations deliver a message targeted to a specific audience. There is a large body of scientific literature indicating that certain methods and strategies are more effective in achieving this goal than others. I have recently participated workshops on science writing and communication, and also did some independent research and reading in the topic. While most of the courses in the science curriculum give a solid foundation regarding the scientific method, very little is taught about the underlying cognitive processes and the `human factor' involved in communicating the results. Such a course would help the students (and the teacher) to become more effective communicators.
\vspace{0.5pc}

