\section{Selected presentations}
%\cventry{when}{Title}{Conf}{Where}{}{extra}

\subsection{Invited}%%----------------------------------
%\subsection{Invited (\FIXME{upcoming talk})}%%----------------------------------

\cvlistitem{2014: {\psolymos} \& MC Roy, ``ABMI: Enhancing our Vision of Alberta's Biodiversity'', invited talk, 2014 Workshop of the Canadian Society of Environmental Biologists (CSEB), Edmonton AB, October 3, 2014}

\cvlistitem{2014: ``Understanding the sources of biases in population size estimates based on roadside surveys in Canada'', invited symposium talk, AOU/COS/SCO Joint Meeting, Estes Park, Colorado, September 23--28, 2014}

\cvlistitem{2014: ``Data cloning: bridging the Bayesian and frequentist statistical paradigms'', guest lecture, Budapest R User Group meetup, Budapest, Hungary, July 16, 2014}

\cvlistitem{2014: ``What to do with messy data: using standardized abundance indices in species distribution modeling'', invited talk, Biometry Conference (Hungarian Society for Clinical Biostatistics), Budapest, Hungary, May 16--17, 2014}

\cvlistitem{2013: ``Complex models in ecology: challenges and solutions'', invited talk, ``Recent advances in R packages'', session organizer: D. Murdoch, Annual Meeting of the Statistical Society of Canada, May 26--29, 2013}

\cvlistitem{2013: ``Parallel computing with R'', guest lecture, Edmonton R User Group meetup, Edmonton, Alberta, April 26, 2013}

%\cvlistitem{2008: ``Distribution data bases and predictive biogeography'', plenary lecture, 3rd Quantitative Ecological Symposium, Budapest, Hungary}

%\cvlistitem{2007: ``The significance of distribution data in conservation biology'', invited talki ``Diversity of diversity measures'', workshop of the Hungarian Biological Society, Budapest, Hungary}

%\cvlistitem{2006 and 2007: ``Biodiversity Assessment'', invited lecture, Applied Ecology Seminar, organized by J.~Padisák \& A.~Liker, University of Pannonia, Veszpr\'{e}m, Hungary}

%\cvlistitem{2006: ``Conservation prospects of Hungarian mollusc species'', invited talk, ``Management of invertebrates in Hungary'' workshop, T\'{u}rkeve, Hungary}

%\cvlistitem{2001: ``Shells and their interpretation in an archaeological context'', invited talk, 5th Workshop of Environmental Archaeology, MATRICA Museum, Sz\'{a}zhalombatta, Hungary}

\subsection{Contributed (last 5 years)}%%--------------
%\subsection{Contributed (last 5 years, \FIXME{upcoming talk})}%%--------------

\cvlistitem{Scarl, J.~C., Lambert, J.~D., Hart, J., Dettmers, R., {\psolymos}~(2014): Ten years of mountain birdwatch: abundance and trends of Bicknell's Thrush and other high-elevation birds. Northeast/Southeast Partners in Flight Bird Conservation Conference, 6--9 October 2014, Wyndham Virginia Beach, VA.}

\cvlistitem{{\psolymos}, Matsuoka, S., Bayne, E., Lele, S.~(2014): Discussing problems vs.~finding solutions: an operational framework for dealing with imperfect detection in species distribution modelling. International Statistical Ecology Conference 2014, Montpellier, France, July 1--4, 2014, p.~135.}

\cvlistitem{Schieck, J., Huggard, D., {\psolymos}~(2014): Biodiversity monitoring to assess cumulative effects: the dichotomy between targeted and surveillence monitoring unraveled (\#541). North America Congress for Conservation Biology (NACCB) to be held in Missoula, Montana, July 13-16, 2014.}

\cvlistitem{Bayne, E., Mahon, L., {\psolymos}, Lankau, H., Ball, J., Tigner, J.~(2014): Integrating uncertainty in edge effects in land-use policy (\#571). North America Congress for Conservation Biology (NACCB) to be held in Missoula, Montana, July 13-16, 2014.}

\cvlistitem{Stralberg, D., Matsuoka, S., Hamann, A., Bayne, E., {\psolymos}, Schmiegelow, F., Wang, X., Cumming, S., Song, S.~(2014): Projecting boreal bird responses to climate change: the signal exceeds the noise (\#270). North America Congress for Conservation Biology (NACCB) to be held in Missoula, Montana, July 13-16, 2014.}

\cvlistitem{{\psolymos}~(2013): Statistical assumptions in the distance decay relationship and their implications for biodiversity conservation. Special Meeting of the International Biogeography Society: The Geography of Species Associations, Montr\'{e}al Qu\'{e}bec, Canada, November 15--17, 2013, p.~30.}

\cvlistitem{Bayne, E., {\psolymos}, Matsuoka, S., Stralberg, D., Fontaine, T., Cumming, S., Schmiegelow, F.~\& Song, S.~(2012): Estimating population sizes of landbirds from non-standardized point-count surveys in North America’s boreal forest: making the most of a potentially messy situation. Abstract Book, 5th North American Ornithological Conference (NAOC--V) in Vancouver, British Columbia, Canada, August 13--19, 2012, p.~52.}

\cvlistitem{Stralberg, D., Bayne, E., Schmiegelow, F., {\psolymos}, Cumming, S., Matsuoka, S., Fontaine, T., \& Song, S.~(2012): Forest passerine distribution models and climate change projections for boreal North America: addressing challenges and uncertainties. Abstract Book, 5th North American Ornithological Conference (NAOC--V) in Vancouver, British Columbia, Canada, August 13--19, 2012, p.~317.}

\cvlistitem{Burton, C., Huggard, D., Schieck, J., {\psolymos}, Bayne, E.~\& Boutin, S.~(2012): A framework for monitoring the cumulative effects of human footprint on biodiversity. Congress Abstracts, The Inaugural SCB North American Congress for Conservation Biology, Oakland, California, July 15--18, 2012,~p.~31.}

\cvlistitem{Mahon, C. L., Bayne, E. M., {\psolymos}, Matsuoka, S. M., Carlson, M.~\& Dzus, E.~(2012): Does expected future habitat condition support proposed population objectives for boreal landbirds in Bird Conservation Region 6 --- Boreal Taiga Plains. ESA 97th Annual Meeting, Portland, Oregon, August 5--10,~2012.}

\cvlistitem{Lele, S.~\& {\psolymos}~(2012): Data cloning based estimability diagnostics and model selection for composite likelihood methods: Theory with application in modeling ecological communities. BIRS Meeting, Banff, AB}

\cvlistitem{{\psolymos}, Bayne, E.~\& Lele, S.~(2010): Correcting biodiversity intactness indices for imperfect detection of species. ICCB 2010, Edmonton, Canada, p.~101.}

%\cvlistitem{{\psolymos}, Lele, S., Moreno, M.~\& Bayne, E.~(2009): Correcting occurrences for detection error: a case study using bird data of the Alberta Biodiversity Monitoring Institute. Society of Canadian Ornithologists - 2009 Annual Conference, August 20--23, 2009, Edmonton, Canada, Program and Abstracts, pp.~29--30.}

