%\section{Statement of teaching philosophy}

An effective teacher is a good communicator who can explain complex concepts in plain terms and who can engage students and maintain interest. An effective teacher tries to find the right balance between providing fundamental knowledge and skills and encouraging students to think critically and independently. An effective teacher should challenge students within reasonable limits, but also provide opportunities for success.
\vspace{0.5pc}

I taught undergraduate level practicals in animal systematics as a teaching assistant at the University of Debrecen between 1998 and 2000. I used a combination of lectures with transparencies and (real) slides, and hands-on identification practice where feasible (insects, molluscs). Teaching assistants were expected to teach basic terminology and working knowledge of concepts in zoology. I found that the often difficult process of memorization could be facilitated by discussing the etymology of the words, for example the name \emph{Neopilina galathea} stems from the species' discovery by the `Galathea' deep-sea expedition and its resemblance to the extinct Paleozoic \emph{Pilina} genus. Teaching assistantships gave me early confidence and experience in teaching.
\vspace{0.5pc}

I continued teaching practicals during my PhD years at the University of Debrecen. I developed the course ``Terrestrial malacology'' that I taught once in each year between 2001 and 2003. I used lectures that combined classical and modern malacological literature with field anecdotes. I think that it is important to make lectures more enjoyable and memorable by telling stories. This is also a way of retaining attention during classes. These lectures helped me develop the ability to respond to interest levels of an audience. I developed the malacology course in the hopes that interesting topics would attract students. As a result, I co-supervised three students with my PhD supervisor Prof.~Z.~Varga. This outcome indicated that I was able to spark enthusiasm and generate interest among the students. Two of the students won a shared prize at a Scientific Student Conference in 2005 with their undergraduate research comparing different field methods. The same work later became a peer-reviewed proceedings paper co-authored with the students (proceedings \#1).
\vspace{0.5pc}

In my undergraduate physics classes we measured Earth's gravity based on Cavendish's torsion-bar experiment in a dark classroom using a LASER beam to trace how the bar rotates. This motivated me to develop hands-on experiments and simulations when I developed my ecology classes at Szent Istv\'{a}n University after in 2004. I team-taught several courses (``Introduction to ecology'', ``Ecology 1'', ``Ecology 2'') with other faculty members. Each course consisted of weekly lectures and weekly/biweekly practicals. I built my lectures from first principles (e.g.~exponential growth, species are independent, individuals are identical) and I encouraged the students to actively participate in guided discussions. These guided discussions led to the modifications of assumptions (growth is limited, species interact, individuals vary), and ultimately to refined models and theory. This way, students had an opportunity to have some `eureka' moments by rediscovering ecology's fundamental principles. I developed practicals to reinforce the knowledge from lectures through active experimentation, because I believe that the best way to teach science is to do science. My course material, for example, using flour beetles to detect density dependence, is still being used at the institute. 
\vspace{0.5pc}

I also led field camps with 7 days of field work in a remote location in Hungary followed by 3 days of lab work  that included statistical analysis of the results (population size estimation, clustering/ordination, indicator species analysis). I found that students worked best in small groups even in the computer lab, because they could split up the tasks among themselves to increase efficiency. In this way they also had time to ponder the topics at hand. This setting facilitated problem solving, critical thinking, and creativity which are essential skills for researchers.
\vspace{0.5pc}

I also developed my own ``Conservation biology'' and ``Zoogeography'' courses. I developed my courses around questions (e.g.~What is biodiversity? What is happening to it? Why do we care? What do we do to protect and restore it?), or well known facts (characteristic organisms and Earth's biomes) to facilitate critical thinking and active engagement through discussions and small student presentations. I also gave guest lectures and coordinated seminars that were team taught (``Frontiers in ecology'', ``Journal club in ecology'') or were based on invited speakers (``National parks of Hungary''). I was teaching approximately 0.4 FTE including grading and examinations. I updated my ``Malacology'' course in 2005, this time extended with a field trip. This period helped me to deepen my knowledge base in general ecology and biodiversity conservation. I used various techniques (slide shows, drawing/writing on white board, real specimens, audio, video) to support the learning success of visual, auditory and tactile learners. I improved my efficiency and organization skills, and participated in the revision of the zoology curriculum at the Institute for Biology (Szent Istv\'{a}n University) for accreditation purposes in 2007 (the new curriculum started in 2008).
\vspace{0.5pc}

I started as postdoc at the University of Alberta in 2008. I have given a few guest lectures each year on various topics, but mostly on statistical subjects (multiple regression, survival analysis, MCMC techniques). I was lucky to have the opportunity to contribute to these courses and maintain a link with teaching. I developed a 1-day short course on the data cloning algorithm used in hierarchical modeling. I participated in seminars on improving written and oral communication based on cognitive science. These seminars helped me reflect on myself, and use communication techniques more purposefully.
\vspace{0.5pc}

Finally, I believe a teacher's job is not finished with teaching \emph{how} to do science. A teacher must demonstrate through examples of their own research and enthusiasm, and should encourage students to think about \emph{why} doing science matters. Above skills and creativity, adopting a mission is increasingly important in our rapidly changing world.
