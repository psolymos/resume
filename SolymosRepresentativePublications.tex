\documentclass[10pt,letterpaper,roman]{moderncv}

\usepackage{SolymosCV}

\title{Recent publications}

\begin{document}

\makecvtitle

\competitionnumber\vspace{4mm}

The three representative publications are intended to show 
(1) the breadth of my research in terms of topics and numerical approaches, 
(2) that I am publishing in leading journals in my fields, 
and (3) that I am involved in highly collaborative research.

\section{S\'{o}lymos et al.~2013}

\psolymos, Matsuoka, S.~M., Bayne, E.~M., Lele, S.~R., 
Fontaine, P., Cumming, S.~G., Stralberg, D., Schmiegelow, 
F.~K.~A. and Song, S.~J. (2013). 
Calibrating indices of avian density from non-standardized survey data: 
making the most of a messy situation. 
\emph{Methods in Ecology and Evolution}, 4:1047--1058.

\cvlistitem{%
I led the team of international collaborators in writing this paper. 
I developed a rigorous statistical estimating procedure based on
original ideas of team members.
Our method accounts for differences in survey protocol
and imperfect detection in a wide array of modeling settings, 
including linear models and machine learning techniques.
The methodology in this paper laid the groundwork for many 
applications that we are currently working on, 
including population size and trend estimation, 
and climate change forecasting at continental scales.
The journal is in the top 10\% of journals in ecology.}

\section{S\'{o}lymos et al.~2012}

\psolymos, Lele, S.~R. and Bayne, E. (2012). 
Conditional likelihood approach for analyzing 
single visit abundance survey data in the 
presence of zero inflation and detection error.
\emph{Environmetrics}, 23:197--205.

\cvlistitem{%
I and S.~Lele conceived the idea and worked together on the mathematics, 
I implemented the computer code 
(available in the \textbf{detect} R extension package). 
E.~Bayne provided the data that I analyzed and  
wrote the first draft. 
We demonstrate that correcting for detection error 
is possible without repeated visits to sites,  thus allowing for 
cost effective monitoring of biodiversity. 
This topic is controversial and it has repeatedly 
been stated in the literature that single visit based 
corrections are not possible. We successfully convinced 
the editors and reviewers that it is indeed possible. 
This journal is one of the leading journals in biostatistics.}

\section{S\'{o}lymos \& Lele 2012}

\psolymos and Lele, S.~R. (2012). 
Global pattern and local variation in species-area relationships. 
\emph{Global Ecology and Biogeography}, 21:109--120.

\cvlistitem{%
I conceived the idea of the paper and compiled 
species-area (SAR) data sets from the literature. 
We refined the modeling technique and wrote the paper together. 
We exploited the methodological innovation (\textbf{dclone} R 
extension package) based on a statistical optimization technique 
called data cloning that I have worked with S.~Lele as a postdoc. 
We showed that the global pattern in SAR is a result of the 
interaction of multiple factors, and that local variation is 
important to consider when making predictions.
The journal is in the top 10\% of journals in ecology and ranked 
1st in physical geography.}

%\subsection{S\'{o}lymos \& Feh\'{e}r 2005}

%\psolymos and Feh\'{e}r, Z. (2005). 
%Conservation prioritization using land snail distribution data in Hungary. 
%\emph{Conservation Biology}, 19:1084--1094.
%
%\cvlistitem{%
%I concieved the idea of the paper, together we 
%have gathered distribution data, I analyzed the data and led
%the writing of the paper.
%In this paper we have assessed the Hungarian land snails from a 
%conservation and biogeographic perspective. 
%The results led to a popular articles, 
%recommendations for species protection, 
%updating of the IUCN RedList status assessment for several Mollusc species. 
%The paper was also a chapter in my PhD thesis. 
%The journal is in the top 15\% of journals in the 
%biodiversity conservation subject area.}

\end{document}

